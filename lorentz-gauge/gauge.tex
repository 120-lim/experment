\documentclass{jarticle}
\usepackage[dvipdfmx]{graphicx}
\usepackage{here}
\usepackage{amsmath}
\usepackage{amsfonts}
\usepackage{listings}
\usepackage{bm}
\title{}
\author{}
\begin{document}
\maketitle

\begin{equation}
\bm{A} = \int d^4r' \frac{\bm{j}(\bm{r'},t)\delta(t'-t-\frac{R}{c})}{|\bm{r}-\bm{r'}|}
\end{equation}
\begin{equation}
\nabla \cdot \bm{A} = \int d^4\bm{r'} \frac{\nabla^\prime\cdot(\bm{j(r',t')}\delta(t'-t_R)) + \bm{j(r',t')}\nabla\cdot\delta}{|\bm{r}-\bm{r'}|}
\end{equation}
ここで、
\begin{equation}
\nabla\cdot\frac{1}{|\bm{r}-\bm{r'}|}=-\nabla^\prime\cdot\frac{1}{|\bm{r}-\bm{r'}|}
\end{equation}
を用いて部分積分をすることで、(2)式の分子第1項を得る。
式(2)の$\delta$の微分の部分がキャンセルして、電流微分のみ生き残る。
\begin{equation}
\phi = \int d^4 r' \frac{\rho(r',t')\delta(t'-t_R)}{|\bm{r}-\bm{r'}|}
\end{equation}
を$t$微分すると、分子には
\begin{equation}
(\frac{\partial \rho(r',t')}{\partial t'})\delta(t'-t_R) =
(\nabla^\prime\bm{j}(r',t'))\delta(t'-t_R)
\end{equation}
が残る。ここで、$\delta$関数に関する以下の公式を用いた。
\begin{equation}
\int \frac{\partial \delta(t'-t)}{\partial t}f(t')dt'=
\int \delta(t'-t)\frac{\partial f}{\partial t'}(t')
\end{equation}
これらの式から以下のLorentz gaugeの式を遅延ポテンシャルは満たしていることがわかる。
\begin{equation}
\nabla\cdot\bm{A} - \frac{1}{c^2}\frac{\partial \phi}{\partial t}=0
\end{equation}
\\

なお、静電場のCoulomb gaugeの確認に関しては、上の計算から$\delta$関数を除去してやれば、
\[ \nabla\cdot \bm{A} = 0\]
を得ることが出来る。

\end{document}
