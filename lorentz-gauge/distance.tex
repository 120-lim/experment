\documentclass{jarticle}
\usepackage[dvipdfmx]{graphicx}
\usepackage{here}
\usepackage{amsmath}
\usepackage{amssymb}
\title{量子力学レポート}
\author{}
\begin{document}
\maketitle

\makeatletter
    \renewcommand{\theequation}{%
    \thesection.\arabic{equation}}
    \@addtoreset{equation}{section}
  \makeatother


距離空間とは、点の集合$M$で、以下の性質を満たす距離関数$d$を持つもの。
\\
$d:M\times M \rightarrow \mathcal{R}$は$M$の点の組$(x,y)$に対し、ある実数を対応させる
関数で、次の(a),(b),(c)を満たすもの。
\\
\begin{equation}
(a):d(x,y) \geq 0 かつ d(x,y)=0 \Leftrightarrow x = y (正値性)
\end{equation}
\begin{equation}
(b):d(x,y) = d(y,x) (対称性)
\end{equation}
\begin{equation}
(c):d(x,z) \leq d(x,y) + d(y,z) (三角不等式)
\end{equation}

例1:ユークリッド距離は明らかに上の3つを満たす。
\\
例2:離散距離を以下のように定義すると、これも3つをみたす。
\\
$d(x,y) = 0$ (when  $x=y)$,
\\
$d(x,y) = 1(otherwise)$ 

\end{document}

 
